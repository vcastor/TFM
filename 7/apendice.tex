\appendix
\renewcommand{\thetable}{SI\arabic{table}}%

\chapter{Supporting Information}

\section{Supporting tables}

{\small{
\begin{tcolorbox}[tab2,tabularx={X||Y|Y|Y},title=Energy for systems at different outputs hartree units ,boxrule=0.5pt]
Cluster    & MP2                & DFT (M06-2X)       & Promelf (M06-2X) \\\hline\hline
dimer      & -152.539151959111  & -152.868036441310  & -152.13643137    \\
trimer     & -228.827663306177  & -229.316617892800  & -225.51627000    \\
tetramer   & -305.359223571261  & -305.763780855504  & -304.27701150    \\
pentamer c & -381.398610398789  & -382.208963954852  & -374.26565610    \\ 
pentamer p & -381.400573846193  & -382.210068626837  & -380.34641670
\label{mp2vsm062x_table}
\end{tcolorbox}

\begin{tcolorbox}[tab2,tabularx={X||Y|Y|Y|Y|Y},title=Energy for the system at wB97XD DSD and M06D3 functionals Hartree unit ,boxrule=0.5pt]
Cluster    & wB97XD      & DSD         & M06D3       & average    & $\sigma$   \\\hline\hline
pentamer p & -382.256096 & -381.796990 & -382.171231 & -382.07477 & 0.24428056 \\
monomer    & -76.439928  & -76.348356  & -76.422501  & -76.403595 & 0.04862536 \\
dimer      & -152.887534 & -152.704601 & -152.852667 & -152.81493 & 0.09712828 \\
trimer     & -229.344559 & -229.069851 & -229.293213 & -229.23587 & 0.14605474 \\
tetramer   & -305.802591 & -305.436006 & -305.732757 & -305.65712 & 0.19464603 \\
pentamer c & -382.257905 & -381.799987 & -382.169303 & -382.07573 & 0.24287605
\label{energies_table}%
\end{tcolorbox}

\begin{tcolorbox}[tab2,tabularx={X||Y|Y},title=Distance average between O atoms in optimized systems Å unit ,boxrule=0.5pt]
  Bond            & Average & $\sigma$\\\hline\hline
  Dimer           & 2.9032 & 0.0052 \\
  Trimer (HB 1)   & 2.7875 & 0.0023 \\
  Trimer (HB 2)   & 2.7965 & 0.0056 \\
  Trimer (HB 3)   & 2.7889 & 0.0024 \\
  Tetramer (HB 1) & 2.7450 & 0.0101 \\
  Tetramer (HB 2) & 2.7423 & 0.0102 \\
  Tetramer (HB 3) & 2.7423 & 0.0055 \\
  Tetramer (HB 4) & 2.7424 & 0.0101
  \label{distances_table}%
\end{tcolorbox}

}}
\newpage
\thispagestyle{empty}

\section{Scripts}\label{SIscript}

All scripts written while we was working on this thesis are upload in a private
GitHub repository
(\href{https://github.com/vcastor/TFMscripts}{github.com/vcastor/TFMscripts}).
The code that not has an execution of {\sc{Promelf}} or use the LCT queue
system is written down below.

Script to write a new wavefunction with the ELF maxima as atoms with charge
zero for the nuclei:

\begin{lstlisting}[language=Python]
#!/opt/homebrew/bin/python3
# write a new wfn file with the coordiantes from a xyz in Atomic Units
import sys 

xyz = sys.argv[1]
oldwfn = sys.argv[2]
newwfn = sys.argv[3]

a = []
x = []
y = []
z = []

archivo = open(xyz, 'r')
datacoor = archivo.readlines()
n = datacoor[0]
n = int(n)

for i in range(n):
	linen = datacoor[i+2]
	linen = linen.split()
	a.append(str(linen[0]))
	x.append(float(linen[1]))
	y.append(float(linen[2]))
	z.append(float(linen[3]))
archivo.close()

archivo = open(oldwfn, 'r')
dataold = archivo.readlines()
header = dataold[1]
header = header.split()
m = header[6]
m = int(m)

with open(newwfn, 'w') as f:
	sys.stdout = f 
	print(dataold[0], end='')
	print(header[0], '{:>14}'.format(header[1]), *header[2:4], '{:>6}'.format(header[4]), header[5], '{:>8}'.format(m+n), header[7])
	for i in range(m):
		print(dataold[i+2], end='')
	for i in range(n):
		print(' ', a[i], ' ', '{:2d}'.format(m+i+1),'   (CENTRE', '{:2d}{}'.format(m+i+1,') '), f'{x[i]:+.8f}', f'{y[i]:+.8f}', f'{z[i]:+.8f}', ' CHARGE =  0.0')
	for i in range(len(dataold)-m-2):
		print(dataold[i+m+2], end='')
\end{lstlisting}

Where the first argument of the last script is the output of:

\begin{lstlisting}[language=Python]
#!/opt/homebrew/bin/python3
import sys

coor = sys.argv[1]
xyz = sys.argv[2]

a = []
x = []
y = []
z = []

archivo = open(coor, 'r')
data = archivo.readlines()
n = data[0]
n = int(n)

for i in range(n):
	linen = data[i+1]
	linen = linen.split()
	a.append(str(linen[2]))
	if (a[i] == "(3,-3)"):
		x.append(float(linen[3]))
		y.append(float(linen[4]))
		z.append(float(linen[5]))
archivo.close()

with open(xyz, 'w') as f:
	sys.stdout = f
	print(len(y),'bohr\n')
	maxELF = True
	i = 0
	while (maxELF):
		if (a[i] == "(3,-3)"):#the xyz file WILL BE IN bohr not angstrom
			print('X', x[i], y[i], z[i])
			i+=1
		else:
			maxELF = False
\end{lstlisting}

The value integrals given by {\sc{Promelf}} were summarized in a smaller
file with the following script:

\begin{lstlisting}[language=perl]
#!/usr/bin/perl5.18 -w
#use strict;
#use warnings;

$outpromelf = $ARGV[0];
$integralsv = $ARGV[1];

my $line=`grep -ne "RELEVANT RESULTS FOR ALL THE ATOMS WITH LMAX = 10" $outpromelf | cut -f1 -d:`;
my $lineInicio=$line + 3;
my @line2=`grep -ne "------------------------------- TES PARTITION ---------------------------------" $outpromelf | cut -f1 -d:`;
my $lineEnd=$line2[1] - 3;
my $d='$d';
system(`sed '1,${lineInicio}d;${lineEnd},$d' $outpromelf > tmp.1`);

my $nmax=`wc -l < tmp.1`;
$nmax=$nmax + 0;
system(`echo $nmax > $integralsv`);
system(`cat tmp.1 >> $integralsv`);
system(`rm tmp.1`)
\end{lstlisting}

And the interactions also given by {\sc{Promelf}} with the next one:

\begin{lstlisting}[language=Python]
#!/opt/homebrew/bin/python3
#run as:
#./interactions.py *.put *.txt

import sys
import os
import numpy as np

pout  = sys.argv[1]
atoms = sys.argv[2]

couint     = []
xcint      = []
kinetic1   = []
kinetic2   = []
potential1 = []
potential2 = []
cou1       = []
cou2       = []
xc1        = []
xc2        = []
max1       = []
max2       = []
DI1        = []
DI2        = []

#################################
# How many interactions between ELF max's or nuclei
cmdcat = "cat " + str(atoms) +  " | sed '/^\\s*\$/d' | wc -l"
n = os.popen(cmdcat).read() 
n = int(n)

archivo2 = open(atoms, 'r')
dataAtoms = archivo2.readlines()
for i in range(n):
	linen = dataAtoms[i]
	linen = linen.split()
	max1.append(int(linen[0]))
	max2.append(int(linen[1]))
archivo2.close()

#################################
# Contributions of Max1 et Max2
cmdgrep = "grep -ne 'Atomic Contributions for neq: ' "+ str(pout) +" | cut -f1 -d:"
outgrep = os.popen(cmdgrep).read()
outgrep = outgrep.split()

archivo1 = open(pout, 'r')
data = archivo1.readlines()
for i in range(n):                               #for max1
	dataT  = data[int(outgrep[max1[i]-1])+2] #kinetic
	dataV  = data[int(outgrep[max1[i]-1])+3] #potential
	dataC  = data[int(outgrep[max1[i]-1])+5] #Cou
	dataXC = data[int(outgrep[max1[i]-1])+6] #XC
	dataT  = dataT.split()
	dataV  = dataV.split()
	dataC  = dataC.split()
	dataXC = dataXC.split()
	kinetic1.append(float(dataT[3]))
	potential1.append(float(dataV[3]))
	cou1.append(float(dataC[2]))
	xc1.append(float(dataXC[2]))
	#                                        #for max2
	dataT  = data[int(outgrep[max2[i]-1])+2] #kinetic
	dataV  = data[int(outgrep[max2[i]-1])+3] #potential
	dataC  = data[int(outgrep[max2[i]-1])+5] #Cou
	dataXC = data[int(outgrep[max2[i]-1])+6] #XC
	dataT  = dataT.split()
	dataV  = dataV.split()
	dataC  = dataC.split()
	dataXC = dataXC.split()
	kinetic2.append(float(dataT[3]))
	potential2.append(float(dataV[3]))
	cou2.append(float(dataC[2]))
	xc2.append(float(dataXC[2]))
archivo1.close()

#################################
#Interaction between Max1 et Max2

cmdgrep = "grep -ne '===================  Interaction with atom:' "+ str(pout) +" | cut -f1 -d:"
outgrep = os.popen(cmdgrep).read()
outgrep = outgrep.split()

#x^2 -x -len(outgrep) = 0; 1+sqrt(1+4*len)/2
natoms = (1+np.sqrt(1+ 4*len(outgrep)))/2
natoms = int(natoms)

for i in range(n):
	if (max1[i] > max2[i]):
		a = outgrep[(max1[i]-1)*(natoms-1) + max2[i] - 1]
	else:
		a = outgrep[(max1[i]-1)*(natoms-1) + max2[i] - 2]
	a = int(a) + 1
	line = data[a]
	line = line.split()
	couint.append(line[4])
	xcint.append(line[5])

cmdcat = "cat " + pout +  " | sed '/^\\s*\$/d' | wc -l"
m = os.popen(cmdcat).read()
m = int(m)
m = m - natoms - 34

for i in range(n):
	dataDI1 = data[int(m+max1[i])-1]
	dataDI2 = data[int(m+max2[i])-1]
	dataDI1 = dataDI1.split()
	dataDI2 = dataDI2.split()
	DI1.append(float(dataDI1[6]))
	DI2.append(float(dataDI2[6]))

for i in range(n):
  print(couint[i], xcint[i], cou1[i], xc1[i], kinetic1[i], potential1[i], DI1[i], cou2[i], xc2[i], kinetic2[i], potential2[i], DI2[i])
\end{lstlisting}

A wrapper was also written to run TopMod or NCI with the same input file.

\begin{lstlisting}[language=Python]
#!/opt/homebrew/bin/python3
#This wrapper launch TopMod and/or NCI with the same input
#I was written with a bash script lol

import sys
import os
from tqdm import tqdm

filein = sys.argv[1]
archivo = open(filein, 'r')
datalines = archivo.readlines()
cmdtop = "/Users/vcastor/Documents/Master_UAM/TFM/scripts/wrapper/topmod.exe "
cmdnci = "/Users/vcastor/Documents/Master_UAM/TFM/nci/nciplot-master/src_nciplot_4.0/nciplot "
cmdstc = "sbf_to_cube "
gridoption = ["fine\n", "ultrafine\n"]

# definimos los definible
def moving(cube):
        cubein = cube.replace(".wfn\n", "_elf.cube")
        cubeout = cube.replace(".wfn\n", "-elf.cube")
        cmd = "mv " + cubein + " " + cubeout
        os.system(cmd)
        cubein = cube.replace(".wfn\n", "_esyn.cube")
        cubeout = cube.replace(".wfn\n", "-esyn.cube")
        cmd = "mv " + cubein + " " + cubeout
        os.system(cmd)

def writetmp(inpf, wfn, inp):
        with open (inpf, 'w') as f:
                f.write(wfn)
                f.write(inp+"\n")
                f.write("1 1 1")

def stc(wfn):
        sbf = wfn.replace(".wfn\n", "_elf.sbf")
        esyn = wfn.replace(".wfn\n", "_esyn.sbf")
        inpsbf = wfn.replace(".wfn\n", ".tmp1")
        inpsyn = wfn.replace(".wfn\n", ".tmp2")
        outf = wfn.replace(".wfn\n", "-stc.out")
        writetmp(inpsbf, wfn, sbf)
        cmd = cmdstc + "< " + inpsbf + " >> " + outf
        os.system(cmd)
        writetmp(inpsyn, wfn, esyn)
        cmd = cmdstc + "< " + inpsyn + " >> " + outf
        os.system(cmd)
        moving(wfn)

def elf(wfn, gridtop):
        outf = wfn.replace(".wfn\n", "_topmodelf.out")
        wfntop = wfn.replace(".wfn\n", ".wfn ")
        cmd = cmdtop + wfntop + str(gridtop) + " &> " + outf
        os.system(cmd)
        stc(wfn)

def nci(wfnorxyz, gridnci):
        inpnci = wfnorxyz.replace(".wfn\n", ".nci")
        outf = wfnorxyz.replace(".wfn\n", "-nciplot.out")
        with open (inpnci, 'w') as f:
                f.write("1\n")
                f.write(wfnorxyz)
                f.write(gridnci)
        cmd = cmdnci + inpnci + " >> " + outf
        os.system(cmd)

def both(wfnorxyz, gridtop, gridnci):
        elf(wfnorxyz, gridtop)
        nci(wfnorxyz, gridnci)

def coffee(array):
        if len(array) > 5:
                c = "I'll take time, you can go for beer"
                c = '{:*^72}'.format(c)
                print(c)
        elif len(array) > 2:
                c = "I'll take time, you can go for coffee"
                c = '{:*^72}'.format(c)
                print(c)

#  Cute header
b = "holi, I hope I'll be useful"
b = '{:*^72}'.format(b)
a = '{:=^72}'.format('=')
print(a)
print(b)
#print(b) ABBA
print(a)

# Grid option, optional
gridtop = '7'        #defoult value 
gridnci = ''         #defoult value 
if (datalines[-1].lower() in gridoption):
        if datalines[-1].lower() == gridoption[0] : gridtop = '8'
        if datalines[-1].lower() == gridoption[1] : gridtop = '9'
        if datalines[-1].lower() == gridoption[0] : gridnci = datalines[-1].upper()
        if datalines[-1].lower() == gridoption[1] : gridnci = datalines[-1].upper()
        datalines = datalines[:-1]

#-----------------
# What will we do?
if (datalines[-2].lower() in {'elf\n', 'nci\n'}):            #launch both
        datalines = datalines[:-2]
        coffee(datalines)
        with tqdm(total=100) as pbar:
                for i in range(len(datalines)):
                        both(datalines[i], gridtop, gridnci)
                        pbar.update(100/len(datalines))
elif datalines[-1].lower() == 'elf\n':                       #launch ELF
        datalines = datalines[:-1]
        coffee(datalines)
        with tqdm(total=100) as pbar:
                for i in range(len(datalines)):
                        elf(datalines[i], gridtop)
                        pbar.update(100/len(datalines))
elif datalines[-1].lower() == 'nci\n':                       #launch NCI
        datalines = datalines[:-1]
        with tqdm(total=100) as pbar:
                for i in range(len(datalines)):
                        nci(datalines[i], gridnci)
                        pbar.update(100/len(datalines))

# A cerrar el chiringuito 
d = 'Bueno, ADIOS'
d = '{:*^72}'.format(d)
print(a)
print(d)
print(a)


# Ahhhhh... delate temporal files
cmd = "rm -f *-stc.out *nci *res *tmp1 *tmp2"
os.system(cmd)
\end{lstlisting}

\newpage
The wapper code can be written with a bash writer that install
TopMod, NCI and even python3 if the computer does not have some of the software required.

\begin{lstlisting}[language=bash]
#!/bin/bash
########################################################################
# Bash wrapper writer. This code look around to know if your computer
# has everything ok to the correct wapper use.
#
#                                                        Victoria Castor
#                          Grupo de Investigación Julia Contreras-García
#                             Paris, Ile de France, Republique Francaise
########################################################################
# Paperback Writer

OS=`uname`                                      # macOS or Linux flavour

# the wrapper was written for python3
# do we have it?
if ! ( command -v python3 &> /dev/null || command -v python &> /dev/null ); then
  if [ "$OS" = 'Darwin' ]; then
    if ! command -v brew &> /dev/null; then
      curl -fsSL https://raw.githubusercontent.com/Homebrew/install/HEAD/install.sh
    fi
    brew install python
  else
    sudo apt-get install python3.8
  fi
fi
# Ok, we have it, but where?
if command -v python3 &> /dev/null; then
    interpreter=( $(type python3) )
    firstline="#!${interpreter[2]}"
    pip3 install tqdm &> /dev/null
elif command -v python &> /dev/null; then
    interpreter=( $(type python) )
    firstline="#!${interpreter[2]}"
    pip install tqdm &> /dev/null
    echo "Be carfule with the python version that you're using"
fi

# One: don't pick up the phone
touch wrapper.py
echo $firstline > wrapper.py
sed -n 2,11p raw_code.txt >> wrapper.py

########################################################################
# Is everything installed for proper wrapper operation?

# TopMod
if [[ ! -f "./topmod.exe" ]]; then
  if ! command -v gfortran &> /dev/null; then
    if [ "$OS" = 'Darwin' ]; then
      if ! command -v brew &> /dev/null; then
        curl -fsSL https://raw.githubusercontent.com/Homebrew/install/HEAD/install.sh
      fi
      brew install gcc
    else
      sudo apt-get install gfortran
    fi
  fi
  gfortran topmod.f90 -o topmod.exe
else
  if [ ! -x "./topmod.exe" ]; then chmod +x ./topmod; fi
fi
echo 'cmdtop = "'`pwd`'/topmod.exe "' >> wrapper.py

# NCI plot
if [[ ! -f "./nciplot" ]]; then
  if ! command -v git &> /dev/null; then
    if [ "$OS" = 'Darwin' ]; then
      if ! command -v brew &> /dev/null; then
        curl -fsSL https://raw.githubusercontent.com/Homebrew/install/HEAD/install.sh
      fi
      brew install git
    else
      sudo apt-get install git
    fi
  else
    git clone https://github.com/juliacontrerasgarcia/nciplot.git
    ( cd ./nciplot/src_nciplot_4.0 ; make mrproper; make )
    echo 'export NCIPLOT_HOME='`pwd`'/nciplot/' >> ~/.bash_profile
    echo 'cmdnci = "'`pwd`'/nciplot/src_nciplot_4.0/nciplot "' >> wrapper.py
  fi
else
  if [ ! -x "./nciplot" ]; then chmod +x ./nciplot; fi
  echo 'cmdnci = "'`pwd`'/nciplot "' >> wrapper.py
fi

# sbf_to_cube
if ! command -v sbf_to_cube &> /dev/null; then
  curl -O https://www.lct.jussieu.fr/pagesperso/silvi/topmod09.tar
  tar -xvf topmod09.tar
  if [ "$OS" = 'Darwin' ]; then
    ( cd ALL ; sed -i '' 's/ifort/gfortran/' 'Makefile' ; make all ; make install )
  else
    ( cd ALL ; sed -i 's/ifort/gfortran/' 'Makefile' ; make all ; make install )
  fi
fi
echo 'cmdstc = "sbf_to_cube "' >> wrapper.py

########################################################################
# c'est fini (about prerequisites)
sed -n '15,$p' raw_code.txt >> wrapper.py

# don't forget give executable permissions
chmod +x wrapper.py

# et voila; danke schon!
echo ""
echo "Everthing done, examples given in the directory: examples"
echo ""
echo "            Grupo de investigación Julia Contreras-García"
echo ""
echo "Read the README.md file to know the wapper limits"
echo ""
echo "Run as:"
echo "./wraper.py input1.inp"

\end{lstlisting}

\newpage

\section{Geometries of water clusters}

Water clusters coordinates in \SI{}{\angstrom}, xyz format:

{\renewcommand{\baselinestretch}{.5}
Set 1:
\\
{\scriptsize{
\noindent \ce{(H_2O)_2} dimer
\VerbatimInput{../clusters/set1/dimer.xyz}
\noindent \ce{(H_2O)_3} trimer
\VerbatimInput{../clusters/set1/trimer.xyz}
\noindent \ce{(H_2O)_4} tetramer
\VerbatimInput{../clusters/set1/tetramer.xyz}
\noindent \ce{(H_2O)_5} pentamer c
\VerbatimInput{../clusters/set1/pentamer_c.xyz}
\noindent \ce{(H_2O)_5} pentamer p
\VerbatimInput{../clusters/set1/pentamer_p.xyz}
}}

\newpage

Set 2:
\\
{\scriptsize{
\noindent \ce{(H_2O)_2} 1071
\VerbatimInput{../clusters/eest/1071.xyz}
\noindent \ce{(H_2O)_2} 2608
\VerbatimInput{../clusters/eest/2608.xyz}
\noindent \ce{(H_2O)_2} 2894
\VerbatimInput{../clusters/eest/2894.xyz}
\noindent \ce{(H_2O)_2} 3427
\VerbatimInput{../clusters/eest/3427.xyz}
\noindent \ce{(H_2O)_2} 3451
\VerbatimInput{../clusters/eest/3451.xyz}
\noindent \ce{(H_2O)_2} 469
\VerbatimInput{../clusters/eest/469.xyz}
\noindent \ce{(H_2O)_2} 5218
\VerbatimInput{../clusters/eest/5218.xyz}
\noindent \ce{(H_2O)_2} 5624
\VerbatimInput{../clusters/eest/5624.xyz}
\noindent \ce{(H_2O)_2} 6003
\VerbatimInput{../clusters/eest/6003.xyz}
\newpage
\noindent \ce{(H_2O)_2} 6280
\VerbatimInput{../clusters/eest/6280.xyz}
\noindent \ce{(H_2O)_2} 7218
\VerbatimInput{../clusters/eest/7218.xyz}
\noindent \ce{(H_2O)_2} 7975
\VerbatimInput{../clusters/eest/7975.xyz}
\noindent \ce{(H_2O)_2} 8332
\VerbatimInput{../clusters/eest/8332.xyz}
\noindent \ce{(H_2O)_2} 8824
\VerbatimInput{../clusters/eest/8824.xyz}
}}
Set 3:
\\
{\scriptsize{
\noindent \ce{(H_2O)_2} 4170%Cs
\VerbatimInput{../clusters/water_clusters/4179_water2Cs.xyz}
\noindent \ce{(H_2O)_3} 4180%UUD
\VerbatimInput{../clusters/water_clusters/4180_water3UUD.xyz}
\noindent \ce{(H_2O)_3} 4181%UUU
\VerbatimInput{../clusters/water_clusters/4181_water3UUU.xyz}
\newpage
\noindent \ce{(H_2O)_4} 4182%S4
\VerbatimInput{../clusters/water_clusters/4182_water4S4.xyz}
\noindent \ce{(H_2O)_4} 4183%Ci
\VerbatimInput{../clusters/water_clusters/4183_water4Ci.xyz}
\noindent \ce{(H_2O)_4} 4184%Py
\VerbatimInput{../clusters/water_clusters/4184_water4Py.xyz}
\noindent \ce{(H_2O)_5} 4185%CYC
\VerbatimInput{../clusters/water_clusters/4185_water5CYC.xyz}
\noindent \ce{(H_2O)_5} 4186%CAA
\VerbatimInput{../clusters/water_clusters/4186_water5CAA.xyz}
\noindent \ce{(H_2O)_5} 4187%CAB
\VerbatimInput{../clusters/water_clusters/4187_water5CAB.xyz}
\noindent \ce{(H_2O)_5} 4188%CAC
\VerbatimInput{../clusters/water_clusters/4188_water5CAC.xyz}
\noindent \ce{(H_2O)_5} 4189%FRA
\VerbatimInput{../clusters/water_clusters/4189_water5FRA.xyz}
\noindent \ce{(H_2O)_5} 4190%FRB
\VerbatimInput{../clusters/water_clusters/4190_water5FRB.xyz}
\noindent \ce{(H_2O)_5} 4191%FRC
\VerbatimInput{../clusters/water_clusters/4191_water5FRC.xyz}
\noindent \ce{(H_2O)_6} 4192%BAG
\VerbatimInput{../clusters/water_clusters/4192_water6BAG.xyz}
\newpage
\noindent \ce{(H_2O)_6} 4193%BK1
\VerbatimInput{../clusters/water_clusters/4193_water6BK1.xyz}
\noindent \ce{(H_2O)_6} 4194%BK2
\VerbatimInput{../clusters/water_clusters/4194_water6BK2.xyz}
\noindent \ce{(H_2O)_6} 4195%CA
\VerbatimInput{../clusters/water_clusters/4195_water6CA.xyz}
\noindent \ce{(H_2O)_6} 4196%CB1
\VerbatimInput{../clusters/water_clusters/4196_water6CB1.xyz}
\newpage
\noindent \ce{(H_2O)_6} 4197%CB2
\VerbatimInput{../clusters/water_clusters/4197_water6CB2.xyz}
\noindent \ce{(H_2O)_6} 4197%CB2
\VerbatimInput{../clusters/water_clusters/4197_water6CB2.xyz}
\noindent \ce{(H_2O)_6} 4198%CC
\VerbatimInput{../clusters/water_clusters/4198_water6CC.xyz}
\noindent \ce{(H_2O)_6} 4199%PR
\VerbatimInput{../clusters/water_clusters/4199_water6PR.xyz}
\newpage
}}
\normalsize
\newpage

