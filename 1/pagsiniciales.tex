\thispagestyle{empty}

\vspace{2cm}
\vspace{1cm}

\begin{tabular}{ll}
Tutora (Sorbonne University) & Dra. Julia Contreras-García\\
Cotutora (Autonomous University of Madrid) & Dra. M. Merced Montero-Campillo\\
 & \\
Jury Members & \\
Chair & Ismanuel Rabadán\\
Secretary & Cristina Sanz \\
 & Paula Pla \\
 & Sandra Rodríguez \\
 & Felipe Zapata 
\end{tabular}

\vspace{4cm}


\begin{multicols}{2}

\begin{center}
--------------------------------------------------\\
Dra. Julia Contreras-García \\
Director\\
\end{center}

\begin{center}
---------------------------------------------------\\
Dra. M. Merced Montero-Campillo\\
Codirector\\
\end{center}

\end{multicols}

\vspace{2cm}

\begin{center}
---------------------------------------------------\\
Victoria Castor Villegas\\
Student\\
\end{center}

\vspace{2cm}

\noindent
This work was done at LCT at Sorbonne University Campus Pierre et Marie Curie.
Paris, Île de France, République française.

\newpage
\thispagestyle{empty}

{\chancery

\textbf{\Huge{Acknowledgements}}

\vspace{1.5cm}

Mis más sinceros agradecimientos a todas aquellas personas que de una u otra
forma han hecho que haya llegado al punto en donde me encuentro hoy.

A la Unión Europea, a través del programa Erasmus Mundus+ que me ha provisto de
lo necesario para realizar mis estudios de máster. Estudios realizados en las
universidades: $i$) Universidad Autónoma de Madrid, $ii$) Université Toulouse
III Paul Sabatier y $iii$) Sorbonne Université Campus Pierre et Marie Curie.
TFM escrito principalmente en ésta última, bajo la supervisión de dos
excelentes asesoras, Julia Contreras-García y Merche Montero-Campillo.

A través de mis 24 años de edad he tenido el apoyo de mi núcleo familiar, con
los que me he formado como persona y siendo un pilar importante para llegar a
ser quien soy. Libia Villegas Morales (madre), Victor Castor Arenas (padre) y
Ana Karen Castor Villegas (hermAna).

Expreso mi gratitud a las amistades que he logrado tener a través del tiempo.
Jenifer Wences,
por tu larga amistad. 
Pau LC,
gracias por ser parte del Manateam \img{1/emojis/Twemoji_1f995}.
%gracias por ser parte del Manateam, Prolina 🦕.
%gracias por ser parte del Manateam, Prolina \emoij{sauropod}.
Andrea FL, 
hail Ciencias.
Angy,
por ser de las mejores entrenadoras Pokémon.

Durante el tiempo vivido en Madrid tuve la grandiosa oportunidad de estar con
los mejores compañeros de piso. Danny, Víctor, José, (\textit{Mauri et Ana as
honorific members}). Pasar el tiempo con ustedes fue de las mejores cosas que
me pudieron haber pasado en este Erasmus. Desde Filomena con nieve en la
cocina, hasta los \SI{40}{\celsius} en pleno verano, pasando por cocinar
%cocina, hasta estar a 40 $^{\circ}$ C en pleno verano, pasando por cocinar
pancakes los findes.

%\begin{CJK*}{UTF8}{bsmi}
%\begin{CJK*}{UTF8}{bkai}
%\begin{CJK*}{UTF8}{gbsn}
\begin{CJK*}{UTF8}{gkai}
非常感谢,高晗。 You smell like a biscuit.
\end{CJK*} 

Compañeros del programa TCCM. El tiempo en Aspet, París, Toulouse y
en Madrid no hubiera sido nada parecido sin ustedes. Intentar esquiar, ir al
Louvre, caminar por calles occitanas, las tarde noches de terraza... gracias.
Sin olvidar a la camarada no inscrita en este máster, pero presente en
casi todo grupo de investigación científica,
%\begin{otherlanguage*}{russian}
\cyrins{\textit{Алекс\'{a}ндра Элбак\'{я}н}}.
%\end{otherlanguage*}


A mis compañeros de investigación: Trinidad, Julen, Bruno y Andrea. A pasar de hablar
distintos dialectos el ``aquario hispanoparlante'' era la mejor oficina
del LCT po'. 

\begin{flushright}
Si al final no me la he pasado tan mal.\\
Como olvidarme de los días en un bar de la Latina.\\
-Ginebras.
\end{flushright}

}

\vspace*{\fill}

